\documentclass[11pt,a4paper]{article}
\usepackage[margin=14mm]{geometry}
\usepackage[T1]{fontenc}
\usepackage[utf8]{inputenc}
\usepackage{lmodern}
\usepackage{paracol}
\usepackage{enumitem}
\usepackage{xcolor}
\usepackage[hidelinks]{hyperref}
\setlength{\parskip}{4pt}
\setlength{\parindent}{0pt}
\definecolor{accent}{HTML}{0EA5E9}
\newcommand{\sectiontitle}[1]{\vspace{2pt}\textbf{#1}\par\vspace{4pt}}
\begin{document}
\thispagestyle{empty}
{\Large\bfseries GRACJAN ZIEMIANSKI}\par
\textcolor{gray}{AI/ML Engineer (Computer Vision) · Fizyka Stosowana}\par
zieemianskigracjan@icloud.com\par +48 511 235 401\par Italy, Torino 10126\par
\vspace{8pt}
\begin{paracol}{2}
\sectiontitle{Work Experience}
\textbf{Junior AI Specialist — Computer Vision}\par\textcolor{accent}{Bright Coders' Factory \& Tauron} \textcolor{gray}{• 2024‑09 – present}\par AI specialist: integracje CV (YOLOv8, DEIM, Detectron2) dla odczytu liczników. mAP50=0.94, mAP50‑95=0.76; edge latency=34 ms (Android NNAPI), CPU=88 ms. API inference (FastAPI/Flask + Docker), med. response −38\%, \textasciitilde 45 req/s (t4g.medium). Flutter jako interfejs.\par\vspace{6pt}
\textbf{Full‑stack Intern (Mobile Dev)}\par\textcolor{accent}{Bright Coders' Factory} \textcolor{gray}{• 2024‑06 – 2024‑09}\par Mobile dev: Flutter (Android/iOS), web (Next.js), trening YOLOv8, budowa API (Flask), testy i debugowanie.\par\vspace{6pt}
\textbf{System and Network Configuration Intern}\par\textcolor{accent}{T.H. ALPLAST} \textcolor{gray}{• 2019‑11 – 2019‑12}\par Konfiguracja systemów IT, dostosowywanie oprogramowania, sieć, bazy danych.\par
\vspace{10pt}
\sectiontitle{Projects}
\textbf{Tauron – web \& mobile}\par AI meter reader: detekcja ROI + OCR 7‑segmentowe; odporność terenowa. mAP50=0.94, mAP50‑95=0.76; edge latency=34 ms; CPU=88 ms; OCR digit=95\%, sequence=92\%.\vspace{6pt}
\textbf{HackYeah2024 – Deepfake Analyser}\par Analiza deepfake; ROC‑AUC=0.93 na kurat. teście; demo hackathon.\vspace{6pt}
\textbf{HackYeah2023 – UniQuestAI}\par Wyszukiwarka ofert studiów z AI.

\switchcolumn
\sectiontitle{Profile}
Inżynier AI/ML z doświadczeniem w Computer Vision i zapleczem fizycznym. Buduję end‑to‑end modele (PyTorch/YOLO/Detectron2), projektuję i wdrażam API inference (Flask/FastAPI, Docker), dbam o dane i metryki. Łączę metody uczenia głębokiego z modelowaniem i weryfikacją eksperymentalną. Interesuję się MLOps i zastosowaniami AI w energetyce oraz projektach badawczych na styku informatyki i fizyki.
\par
\vspace{6pt}
\sectiontitle{Skills}
\begin{itemize}[leftmargin=*]
\item Python
\item PyTorch
\item Detectron2
\item YOLOv8
\item OpenCV
\item NumPy
\item Pandas
\item scikit‑learn
\item FastAPI
\item Flask
\item Docker
\item SQL / MySQL
\item Label Studio
\item Git
\item Linux
\item CUDA
\item Next.js
\item Flutter
\end{itemize}
\vspace{6pt}
\sectiontitle{Education}
\textbf{Applied Computer Science and Measurement Systems}\par University of Wroclaw (Bachelor) \textcolor{gray}{• 2022 – 2026}\vspace{6pt}
\textbf{MOST Student Exchange – Computer Science}\par AGH University of Krakow (Bachelor) \textcolor{gray}{• 2025‑09 – 2026‑02}\vspace{6pt}
\textbf{ERASMUS – Applied Physics (UniTo)}\par University of Turin (Bachelor (exchange)) \textcolor{gray}{• 2024‑09‑20 – 2025‑02‑28}
\vspace{6pt}
\sectiontitle{Languages}
Polski — Ojczysty\,\, Angielski — B2/C1\,\, Niemiecki — A2 (w trakcie nauki)\,\, Włoski — A2 (w trakcie nauki)

\end{paracol}
\newpage
\sectiontitle{More Experience}
\textbf{President – Pointer UWR (Student Society)}\par\textcolor{accent}{University of Wroclaw} \textcolor{gray}{• 2023‑10 – present}\par Organizacja spotkań, wymiana wiedzy, prowadzenie projektów IT.\par
\vspace{10pt}
\sectiontitle{Certificates}
\textbf{Deep Learning Onramp}\par MathWorks \hfill 2024‑02‑04\vspace{4pt}
\textbf{Powerboat Driver License}\par PZMiNW \hfill 2022‑08‑25\vspace{4pt}
\textbf{Water Skiing Towboat Operator License}\par PZMiNW \hfill 2022‑08‑25\vspace{4pt}
\textbf{EE.08 EE.09 – IT Technician}\par CKE \hfill 2022‑06‑30\vspace{4pt}
\textbf{Polish Matura Exam}\par CKE \hfill 2022‑06‑30\vspace{4pt}
\textbf{Driving License B}\par PL \hfill 2021‑10‑08
\vspace{10pt}
\sectiontitle{Interests}
\textbf{Creating design}\par Stylowo i użytecznie. Poprawiam świat dzięki designowi.\vspace{4pt}
\textbf{Books \& Movies}\par Uwielbiam sci‑fi i historię; pozwala odpocząć i uczyć się.\vspace{4pt}
\textbf{Taekwondo – ITF}\par Dyscyplina, wytrwałość, szacunek – moja filozofia życia.
\vfill\small\textcolor{gray}{Wyrażam zgodę na przetwarzanie moich danych osobowych w celu prowadzenia procesu rekrutacji, zgodnie z RODO.}
\end{document}
